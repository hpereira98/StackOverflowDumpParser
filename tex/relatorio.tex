%% RELATÓRIO DO PROJETO EM C DE LI3

\documentclass[a4paper, 11pt, oneside]{article}

\usepackage[utf8]{inputenc}
\usepackage[T1]{fontenc} 
\usepackage[portuguese]{babel}
\usepackage{graphicx}
\usepackage{hyperref}
\usepackage{xcolor}
\usepackage{multicol}

%% DEFINIÇÃO DOS SNIPPETS EM C
\usepackage{listings}
\lstdefinestyle{customc}{
  belowcaptionskip=1\baselineskip,
  breaklines=true,
  language=C,
  showstringspaces=false,
  basicstyle=\footnotesize\ttfamily,
  keywordstyle=\bfseries\color{green!40!black},
  commentstyle=\itshape\color{purple!40!black},
  identifierstyle=\color{blue},
  stringstyle=\color{orange},
}

\lstset{escapechar=@,style=customc}


%%RELATÓRIO

\begin{document} 

\begin{titlepage}


	\centering 
	
	\scshape
	
	\vspace*{\baselineskip}
	
	
	\rule{\textwidth}{1.6pt}\vspace*{-\baselineskip}\vspace*{2pt}
	\rule{\textwidth}{0.4pt}
	
	\vspace{0.75\baselineskip}
	
	{\LARGE RELATÓRIO DO PROJETO EM C} 
	
	\vspace{0.75\baselineskip} 
	
	\rule{\textwidth}{0.4pt}\vspace*{-\baselineskip}\vspace{3.2pt}
	\rule{\textwidth}{1.6pt}
	
	\vspace{2\baselineskip}
	
	Laboratórios de Informática III
	
	\vspace*{3\baselineskip}
	
	Grupo 24:
	
	\vspace{0.5\baselineskip}
	
	{\scshape\Large Henrique Pereira (a80261) \\ Pedro Moreira (a82364) \\ Pedro Ferreira (a81135) \\}
	
	\vspace{0.5\baselineskip}
	
	\textit{Universidade do Minho \\ Mestrado Integrado em Engenharia Informática}
	
	\vfill

	\includegraphics[width=40mm]{logoUM.jpg}

	\vspace{0.3\baselineskip}
	
	2018

\end{titlepage}

%% ÌNDICE

\tableofcontents
\lstlistoflistings

\newpage

%% INTRODUÇÃO

\section{Introdução}

No âmbito da unidade curricular \underline{Laboratórios de Informática III}, do 2º ano do Mestrado Integrado em Engenharia Informática, foi-nos proposta a realização de um projeto. Projeto este que consistia no desenvolvimento de um sistema capaz de processar ficheiros XML, armazenando as várias informações utilizadas pelo Stack Overflow (uma das comunidades de perguntas e respostas mais utilizadas atualmente por programadores em todo o mundo). Além disso, após o tratamento da informação, o sistema teria que ser capaz de responder eficientemente a um conjunto de interrogações (explicitado na secção \ref{queries}). Esta aplicação teria que ser obrigatoriamente desenvolvida em \textit{C}.

Ora, perante este enunciado, decidimos utilizar na estruturação do nosso sistema as definições da biblioteca \href{https://developer.gnome.org/glib}{GLib}, mais propriamente as GHashTables, as GTrees e os GArrays (e respetivas funções). As estruturas por nós definidas serão apresentadas e justificadas na secção \ref{structs}.

O grupo decidiu organizar o trabalho, separando o código em diferentes ficheiros, isto é, um ficheiro para as estruturas por nós definidas e utilizadas (\textit{structs.c}) e respetivos acessos (getters e setters, como forma de garantir o encapsulamento), um para o parser do ficheiro XML (\textit{load.c}), um para cada interrogação (\textit{query\_*.c}), um para as funções auxiliares (\textit{my\_funcs.c}) e outro para a \textit{main.c}.

Um ponto fulcral deste trabalho era conciliar a eficiência com o encapsulamento, sem comprometer nenhum destes.

%% ESTRUTURAS

\newpage
\section{Estrutura}
\label{structs}

Para a realização do projeto, tivemos que definir vária estruturas de dados em \textit{C}, de maneira a respeitarmos o enunciado, no que toca aos tipos concretos e abstratos de dados. Foi-nos dada a seguinte definição abstrata de uma comunidade do Stack Overflow:

\begin{lstlisting}[caption=Definição da TAD\_community]
typedef struct TCD_community * TAD_community;
\end{lstlisting}

Assim sendo, tivemos como desafio criar a nossa própria definição concreta da comunidade. Após ponderarmos em grupo, tendo em conta as interrogações que nos apresentaram, a eficiência e o encapsulamento, decidimos organizar a nossa \textit{TCD\_community} da seguinte maneira (utilizando as definições da biblioteca GLib):
\begin{itemize}
\item Uma tabela de Hash para os Utilizadores
\item Uma tabela de Hash para informações auxiliares sobre os Posts
\item Uma tabela de Hash para as Tags
\item Uma árvore para os Posts
\end{itemize}

Ou seja, o código para tal é o seguinte:
\begin{lstlisting}[caption=Definição da TCD\_community]
struct TCD_community{
	GHashTable* user;
	GTree* post;
	GHashTable* postAux;
	GHashTable* tags;	
};
\end{lstlisting}

Por sua vez, definimos também uma estrutura para os Utilizadores, para os Posts e para as Tags, com o intuito de as "guardar" em cada elemento das tabelas/árvore acima descritas.

\begin{lstlisting}[caption=Definição de estruturas internas]
struct user{
	long id; // id do utilizador
	char* display_name; // username
	int n_perguntas; // numero de perguntas do utilizador
	int n_respostas; // numero de respostas do utilizador
	int n_posts; // numero total de posts deste utilizador
	int reputacao; // reputacao
	char* short_bio; // descricao do utilizador
	GArray* userPosts; // array com os posts de cada utilizador

};

struct post{
	long id; // id do post
	char* titulo; // titulo do post
	long owner_id; // id do criador do post
	int owner_rep; // reputacao do criador do post
	int type_id; // tipo do post
	long parent_id; // "pai" do post (se este for resposta)
	char* data; // data do post
	GArray* tags; // tags do post
	int score; // score do post
	int n_comments; // numero de comentarios do post
	int n_respostas; // numero de respostas do post (se este for pergunta)
};

struct tag{
	char* name; // nome da tag
	long id; // id da tag
	int ocorrencias; // ocorrencias da tag
};
\end{lstlisting}

%% QUERIES
\newpage
\section{Interrogações e abordagem}
\label{queries}


\subsection{Info From Post}

\subsection{Top Most Active}

\subsection{Total Posts}

\subsection{Questions with Tag}

\subsection{Get User Info}

\subsection{Most Voted Answers}

\subsection{Most Answered Questions}

\subsection{Contains Word}

\subsection{Both Participated}

\subsection{Better Answer}

\subsection{Most Used Best Rep}

%% CONCLUSÃO
\newpage
\section{Conclusão}

\begin{lstlisting}[caption= exemplo código]
#include <stdio.h>
#include <stdlib.h>

//codigo

void exemplo () {
for (int i=0; i<1904; i++) printf("ola\n");
typedef struct Swag = malloc(sizeof(int)*N);
return;
}
\end{lstlisting}

\end{document}
